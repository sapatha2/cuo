\documentclass{article}
\usepackage{graphicx}
\usepackage{xcolor}
\usepackage{subcaption}
\usepackage[margin=1.0in]{geometry}
\usepackage{float}

\begin{document}
\section{Introduction}
\begin{enumerate}
\item \textbf{General goal: }
Systematically develop model Hamiltonians which can accurately describe the energies and properties of the lowest lying eigenstates for transition-metal oxide (TMO) molecules.

\item \textbf{Barrier to achieving that goal: } The necessity for many-body interactions in describing the low-lying excited states of TMO molecules poses a difficult challenge when developing model Hamiltonians.

\item \textbf{State of the art: } Common approaches to building these model Hamiltonians use effective single particle theories like Density Functional Theory (DFT) to generate an effective 1-body model or use the known energy spectrum of the TMO molecules to fit a model with many-body interactions. 
\textcolor{red}{Have to find references for this one, not sure how true this is.}

\item\textbf{How are we advancing state of the art: } We have systematically developed a many-body model which can accurately describe energies and properties of the lowest lying eigenstates for the neutral CuO molecule using \textit{ab-initio} Quantum Monte Carlo (QMC) calculations and a density-matrix downfolding (DMD) procedure. 
\end{enumerate}

\section{Methods}
\begin{enumerate}
\item \textbf{Density matrix downfolding} The DMD procedure allows us to develop low-energy effective theories for physical systems in a systematic manner beginning with the \textit{ab-initio} Hamiltonian $H_{ab}$.

\item \textbf{Defining the low-energy space} We define the low-energy space for the CuO molecule as the full space of states with 9 to 10 electrons in the Cu 3d orbitals.

\item \textbf{Sample states} States within our low energy space (LE) were generated by sampling the span of base states - states which approximately describe true eigenstates within LE.

\item \textbf{Model basis} We constructed two bases for our model Hamiltonian, a localized atomic basis for evaluating two-body reduced density matrix (2-rdm) elements, and a molecular orbital basis for one-body reduce density matrix (1-rdm) elements.

\item \textbf{Model selection and fitting}
\begin{enumerate}
\item We initially narrow down the space of potential models by discarding any models which do not contain the descriptors $n_{4s}, n_{2p_\pi}, n_{2p_z}, J_{sd}, U_s$.

\item Our inability to sample certain states within the LE space manifests itself through intruder eigenstates within our selected models.

\item In order to ensure that intruder states are not present within our final model, we enforce a prior that pushes intruder states above a chosen $E_{cut}$ by altering the cost function for our regression. 

\item \textcolor{red}{Might want to put this in discussion, not sure.}
\end{enumerate}
\end{enumerate}

\section{Discussion}
\begin{enumerate}
\item \textbf{Regression with prior} Using our new cost function we find a set of models which accurately describe our sample set without the presence of intruder states.

\item \textbf{Comparison to experiment} The selected low-energy model accurately describes the energies and single particle properties of the low-energy eigenstates of CuO as measured in experiment.

\item \textbf{Comparison to DFT} Our model is more accurate than a single-particle model constructed using density functional theory orbitals and eigenvalues.
\end{enumerate}

\section{Conclusion}
\begin{enumerate}
\item \textbf{Our findings} We have systematically developed a low-energy model for the CuO molecule which can quantitatively accurately describe the energies and properties of the lowest lying eigenstates as seen in experiment starting from the many-body \textit{ab-initio} Hamiltonian.

\item \textbf{Avenues of interest} The procedure outlined here can be extended to other neutral transition metal oxide molecules, allowing for the construction of a sequence of low-energy models across the transition metals.
\end{enumerate}

%Figure 1 - Show extrapolation error somehow
%Figure 2 - Show prior regression results
%Figure 3 - Show final regression + ED
\end{document}