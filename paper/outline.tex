\documentclass{article}
\usepackage{xcolor}
\begin{document}
\section{Introduction}
\begin{enumerate}
\item \textbf{General goal: }
Systematically develop model Hamiltonians which can accurately describe the energies as well as one- and two-particle properties of the lowest lying eigenstates for transition-metal oxide (TMO) molecules.

\item \textbf{Barrier to achieving that goal: } The necessity for many-body interactions in describing the low-lying excited states of TMO molecules poses a difficult challenge when developing model Hamiltonians.

\item \textbf{State of the art: } Common approaches to building these model Hamiltonians use effective single particle theories like Density Functional Theory (DFT) to generate an effective 1-body model or use the known energy spectrum of the TMO molecules to fit a model with many-body interactions. 
\textcolor{red}{Have to find references for this one, not sure how true this is.}

\item\textbf{How are we advancing state of the art: } We have systematically developed a many-body model which can accurately describe the energies, one- and two-particle properties of the lowest lying eigenstates for the neutral CuO molecule using \textit{ab-initio} Quantum Monte Carlo (QMC) calculations and a density-matrix downfolding (DMD) procedure. 
\end{enumerate}

\section{Figures}

\section{Supplementary Material} 

\end{document}