\documentclass{article}
\usepackage{graphicx}
\usepackage{xcolor}
\usepackage{subcaption}
\usepackage[margin=1.0in]{geometry}
\usepackage{float}
\usepackage{ulem}
\usepackage{amsmath}
\usepackage{mathtools}

\begin{document}
\section{Introduction}

\section{Methods}
The DMD procedure \textbf{[ref]} allows us to develop low-energy effective theories for physical systems in a systematic manner beginning with the \textit{ab-initio} Hamiltonian under a Born-Oppenheimer approximation:

\begin{equation}
H_\text{ab} = -\frac{1}{2} \sum_{i} \nabla_i^2 - \sum_{i,I}\frac{Z}{|r_i - R_I|} + \sum_{i<j}\frac{1}{|r_i - r_j|}
\end{equation}
where $i$ indicates electrons and $I$ nuclei. 
The unit of energy using this Hamiltonian is Hartree (Ha). 
The goal of DMD is to downfold the \textit{ab-initio} Hamiltonian and Hilbert space onto an effective Hamiltonian acting on a low-energy subspace of the Hilbert space 

\begin{equation}
(H_\text{ab}, \mathcal{H}) \xrightarrow{\text{DMD}} (H_\text{eff}, \mathcal{LE} \subset \mathcal{H}) \text{ so that }
\forall |\Psi\rangle \in \mathcal{LE}, \ \langle \Psi | H_\text{ab} | \Psi\rangle = \langle \Psi | H_\text{eff} | \Psi\rangle
\end{equation} 
Broadly the DMD method consists of four steps which will be discussed in detail later: 1) Selecting a low-energy subspace $\mathcal{LE}$ which includes all excitations one wants to describe accurately. 
In theory $\mathcal{LE}$ would be chosen to be the subspace spanned by the N lowest-energy eigenstates of $H_\text{ab}$, but is constructed more approximately in practice.  
2) Generate a set of samples within our low-energy space $\mathcal{SS} \in \mathcal{LE}$. 
The choice of sampling scheme may be system dependent. 
We will discuss a shell-sampling scheme in this paper. 
It is important to note that one does not need to have access to the eigenstates of $H_\text{ab}$ in order to sample a chosen $\mathcal{LE}$, nor are the eigenstates required to be in the sample set.
3) Parameterizing $H_\text{eff}$ as a linear combination of descriptors, 
\begin{equation}
H_\text{eff} = \sum_k a_k \hat{d}_k
\end{equation}
where $\hat{d}_k$ are typically 1- or 2-body reduced density matrix (RDM) elements which can represent single- or two-particle terms in an effective theory.
4) Using the samples to fit the coefficients $\{c_k\}$ such that $H_\text{eff}$ satisfies the conditions in equation (2) as well as possible. 
The fitting procedure is typically an ordinary linear regression, however one may use a custom cost function in the regression as well. 
Therefore the DMD procedure allows us to map the problem of downfolding strongly correlated electrons systems onto a linear regression problem which can then be systematically studied using tools developed in data science and statistics.

Based on anion photoelectron spectroscopy (APES) experiments\textbf{[ref]} on the CuO molecule we define our low-energy space $\mathcal{LE}$ as the space of states with 9 to 10 electrons in the Cu 3d orbitals. 
The APES measurements indicate that the lowest energy excitations in the CuO molecule are between the O p-orbitals and Cu 4s orbitals, leaving around 10 electrons in the Cu 3d orbitals. 
Bonding between the O p and Cu 3d orbitals lead to significant variations in the Cu 3d occupation, indicating the importance of including excitations out of the Cu 3d orbitals. 
This is corroborated by APES measurements indicating that the lowest energy eigenstates with 9 electrons in the Cu 3d orbitals are only $\sim$ 0.2 eV higher in energy than the second excited state of the system with 10 electrons in the Cu 3d orbitals. 
Our particular focus will be on describing most accurately the properties and energies of eigenstates up to 2eV, as higher energy states form a nearly continuous block of states which include vibrational modes.

Our sample space $\mathcal{SS}$ was generated by sampling the span of base states - states which approximately describe true eigenstates within the $\mathcal{LE}$ - using multi-Slater-Jastrow trial wave functions in fixed-node Diffusion Monte Carlo (FN-DMC). 




\section{Discussion}


\section{Conclusion}

\end{document}