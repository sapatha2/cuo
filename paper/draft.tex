\documentclass{article}
\usepackage{graphicx}
\usepackage{xcolor}
\usepackage{subcaption}
\usepackage[margin=1.0in]{geometry}
\usepackage{float}
\usepackage{ulem}
\usepackage{amsmath}
\usepackage{mathtools}

\begin{document}
\section{Introduction}

\section{Methods}
The DMD procedure allows us to develop low-energy effective theories for physical systems in a systematic manner beginning with the \textit{ab-initio} Hamiltonian:
\begin{equation}
H_\text{ab} = -\frac{1}{2} \sum_{i} \nabla_i^2 - \sum_{i,I}\frac{Z}{|r_i - R_I|} + \sum_{i<j}\frac{1}{|r_i - r_j|}
\label{eq:Hab}
\end{equation}
where $i$ indicates electrons and $I$ nuclei, and the unit of energy is Hartree (Ha).
In this paper we will be working under the Born-Oppenheimer approximation and core electrons will be treated using pseudo-potentials.
The procedure defines a method for downfolding the \textit{ab-initio} energy functional and Hilbert space onto an effective energy functional defined on a low-energy subspace of the Hilbert space 
\begin{equation}
(E_\text{ab}, \mathcal{H}) \xrightarrow{\text{DMD}} (E_\text{eff}, \mathcal{LE} \subset \mathcal{H}), \text{ s.t. }
\forall |\Psi\rangle \in \mathcal{LE}, \ E_\text{eff}[\Psi] - E_\text{ab}[\Psi] = \epsilon[\Psi] + E_0.
\label{eq:DMD}
\end{equation} 
The energy functionals can be written as expectations over Hamiltonian operators $E_\text{ab} = \langle \Psi | H_\text{ab} |\Psi \rangle$, $E_\text{eff} = \langle \Psi | H_\text{eff} |\Psi \rangle$, $\epsilon$ represents the error in our effective theory, and $E_0$ is a constant energy shift.
Broadly the DMD method consists of four steps which will be discussed in detail later: 
A) Defining the space of low-energy excitations which $E_\text{eff}$ should be able to describe accurately.
B) Sampling, at minimum, a set of states in $\mathcal{H}$ which probe the chosen low-energy excitations in a statistically independent manner. The selected states need not be eigenstates. 
C) Constructing a set of candidate models which are linear in reduced density matrix (RDM) descriptors with variable coefficients $\{c\}$
\begin{equation}
E_\text{eff} = \sum_k c_k d_k[\Psi]\ + E_0,\ d_k[\Psi] = \langle \Psi | \hat{d}_k |\Psi \rangle.
\label{eq:Eeff}
\end{equation}
Here $\hat{d}_k$ are Hermitian operators typically composed of 1- or 2-RDM elements. 
Examples of possible $\hat{d}_k$ are number operators $\hat{n}$ and exchange operators $\vec{S}_i \cdot \vec{S}_j$.
Additionally, a single particle basis must be built to express RDM elements on and should be chosen such that variations among states in $\mathcal{LE}$ are mostly described by variations in the projection of these states onto the basis.
D) Using the samples to fit the coefficients $\{c\}$ by linear regression such that $E_\text{eff}$ satisfies the conditions in \eqref{eq:DMD} with the smallest error $\epsilon$ possible. 
The fitting procedure is typically an ordinary linear regression, however, one may use a custom cost function in the regression as well. 
In order to conduct the linear regression for a functional like \eqref{eq:Eeff}, one needs to calculate for each state sampled from $\mathcal{LE}$ the expectation values of $H_\text{ab}$ and every descriptor $\{\hat{d}_k\}$ used in the regression.
In principle, if conducted with no approximations, the DMD method is rigorously guaranteed to obtain exact results, namely $\epsilon[\Psi] = 0$.

\pagebreak
peepoo

\pagebreak

Based on anion photoelectron spectroscopy (APES) experiments\textbf{[ref]} on the CuO molecule we define our low-energy space $\mathcal{LE}$ as the space of states with 9 to 10 electrons in the Cu 3d orbitals. 
Mathematically we can express this space as the span of some eigenstates within our Hilbert space
\begin{equation}
\mathcal{LE} = \text{Span(}\{ |\Psi \rangle | \langle \Psi | n_{3d} | \Psi \rangle \ge 9,\ H|\Psi\rangle = E |\Psi\rangle \}\text{)}.
\end{equation}
The APES measurements indicate that the lowest energy excitations in the CuO molecule are between the O p-orbitals and Cu 4s orbitals, leaving around 10 electrons in the Cu 3d orbitals. 
Bonding between the O p and Cu 3d orbitals lead to significant variations in the Cu 3d occupation, indicating the importance of including excitations out of the Cu 3d orbitals. 
This is corroborated by APES measurements indicating that the lowest energy eigenstates with 9 electrons in the Cu 3d orbitals are only $\sim$ 0.2 eV higher in energy than the second excited state of the system with 10 electrons in the Cu 3d orbitals. 
Our particular focus will be on describing most accurately the properties and energies of eigenstates up to 2eV, as higher energy states form a nearly continuous block of states which include vibrational modes.

Our sample space $\mathcal{SS}$ was generated by sampling the span of base states - states which approximately describe true eigenstates within the $\mathcal{LE}$ - using multi-Slater-Jastrow trial wave functions in fixed-node Diffusion Monte Carlo (FN-DMC). In principle we could sample $\mathcal{LE}$ by taking linear combinations of the eigenvectors which span it. In practice we do not have access to these eigenvectors and approximate methods are required to sample the low-energy space. We consider wave functions of the following form: 
\begin{equation}
\lim_{\tau \rightarrow \infty} e^{-\tau H_\text{ab}} (e^{J}\sum_{i} c_i|\text{D}_i\rangle) = \sum_i c_i \lim_{\tau \rightarrow \infty} e^{-\tau H_\text{ab}} (e^J |D_i\rangle) \equiv \sum_i c_i |\Phi_i\rangle .
\end{equation}
where $|D_i\rangle$ is a determinant of single particle orbitals, $e^J$ is a three-body Jastrow function in \textbf{ref}, and $e^{-\tau H_\text{ab}}$ is a projection operator which exponentially suppresses high-energy contributions to the wave function and is realized through an FN-DMC calculation.
This state is an accurate approximation to a state $|\Psi\rangle \in \mathcal{LE}$ of the form $|\Psi \rangle = \sum_i c_i |\Psi_i\rangle$, where $|\Psi_i\rangle$ is an eigenstate within $\mathcal{LE}$, if $\forall i\ |\Phi_i\rangle \sim |\Psi_i\rangle$. While $|\Psi_i\rangle$ are eigenstates of $H_\text{ab}$, $|\Phi_i\rangle$ generally are not, but do form the basis for our sample space and will be called \textit{base states}. The fixed-node projection can achieve this similarity if the nodes of $|D_i\rangle$ accurately approximate those of $|\Psi_i\rangle $. We use symmetry-targeted unrestricted Kohn-Sham (UKS) to generate our determinants $|D_i\rangle$ using a Trail-Needs pseudopotential, VTZ Trail-Needs basis \textbf{ref}, and a B3LYP functional \textbf{ref}, which has been shown to accurately reproduce nodal properties of transition-metal oxide molecules \textbf{ref}. These calculations were carried out in the package PySCF \textbf{ref}. The symmetry-targeted UKS method allows us to fix the total spin projection S$_z$ and the total number of electrons in a particular irreducible representation (irrep) of the symmetry group under use, $\text{A},\ \text{E}_\text{1x},\ \text{E}_\text{2y}$ etc. There are some cases of two or more base states which have identical S$_z$ and number of electrons per irrep. within  $\mathcal{LE}$, for example the ground state and the excited state with excitation $c^\dagger_{p_\pi} c_{d_\pi} |GS\rangle$. In this scenario the higher energy excitations are inaccessible and excluded from our sample set. We will discuss the consequences of this exclusion later on. The three-body Jastrow factor was optimized on the lowest energy UKS state using a linear energy optimization methods \textbf{ref}. The FN-DMC was conducted with T-moves \textbf{ref} to make the calculation variational while using pseudopotentials with a timestep of $\tau = 0.01$. Necessary 1- and 2-RDM elements were also calculated in FN-DMC using a mixed estimator \textbf{ref}. The coefficients $\{c_j\}$ were chosen via a shell-sampling method. For each base state $|\Phi_i\rangle$ we fix the coefficient $c_i = \sqrt{w},\ w \in \{0.2, 0.4, ..., 1.0\}$ and for each $w$ sample N = 5 states uniformly such that $\sum_j c_j^2 = 1$. Looping over $i$ yields shells of samples centered on each base state. This sampling scheme tends to be sparse where the density of states of $H_\text{ab}$ is low and therefore extra samples are generated in regions of sparse sample density \textit{post-hoc}. This sampling scheme therefore generates a sample set which can be defined as 
\begin{equation}
\mathcal{SS} \in \text{Span(}\{ |\Phi \rangle | \langle \Phi | n_{3d} | \Psi \rangle \ge 9,\ H|\Phi\rangle = E |\Phi\rangle \}\text{)} \sim \mathcal{LE}.
\end{equation}
where $|\Phi\rangle$ is defined in (5), and which maps exactly onto the true $\mathcal{LE}$ if our base states exactly match the true eigenstates of $H_\text{ab}$.

Parameterizing our effective model requires constructing a basis on which to calculate our \textit{ab-initio} 1-/2-RDM elements and selecting a set of descriptors from which we would like to build our model. We constructed two bases for our model Hamiltonian, a localized intrinsic atomic orbital (IAO) basis for evaluating two-body reduced density matrix (2-rdm) elements, and a molecular orbital (MO) basis for one-body reduced density matrix (1-rdm) elements. The two-body interactions we are interested in are typically local interactions, for example a Hubbard-U, Hund's-J or superexchange-J, therefore motivating a local basis for the two-body terms. In the CuO molecule there is heavy hybridization between the Cu d and O p orbitals, and an MO basis can compactly represent this hybridization, leading us to use MO basis elements for our one-body basis elements. Based on experimental measurements and the structure of the data within our $\mathcal{SS}$, the lowest energy excitations of the CuO molecule have excitations between Cu d, O p and Cu 4s orbitals. APES measurements do not see any 4s$^2$ states in the low-energy spectra which is corroborated by our FN-DMC energy for the lowest energy base state with a Cu 4s$^2$ occupation is $\sim$ 7 eV above the lowest energy base state. Including the $\sim$ 0.5eV Hund's coupling on the copper atom, we believe that a minimal model for our downfolding must contain: $\bar{n}_{4s}, \bar{n}_{p_\pi}, \bar{n}_{p_z}, \sum_{i \in \{xy, xz, ...\}}\vec{S}_{4s}\cdot \vec{S}_{d_i}, n_{4s, \uparrow} n_{4s,\downarrow}$. The barred terms are built on the MO basis, and the non-barred ones on the IAO basis. The exclusion of $n_{3d}$ from our minimal model is because of the total number of electrons being constrainted by $\bar{n}_{3d} + \bar{n}_{p_\pi} + \bar{n}_{p_z} + \bar{n}_{4s} = 15$, leaving one of these four occupation numbers linearly dependent on the rest. Excited states in CuO have heavy orbital relaxation, so including extra hoppings may be useful. Therefore we define the following model space $\mathcal{MS}$, from which will we try and select the \textit{best} low-energy model for CuO: 

\begin{equation}
\mathcal{MS} = \{\bar{n}_{4s}, \bar{n}_{p_\pi}, \bar{n}_{p_z}, \sum_{i \in \{xy, xz, ...\}}\vec{S}_{4s}\cdot \vec{S}_{d_i}, n_{4s, \uparrow} n_{4s,\downarrow}\} + \text{P}(\bar{t}_\pi, \bar{t}_{dz}, \bar{t}_{sz}, \bar{t}_{ds})
\end{equation}
where P denotes the power set.

After fitting each potential model using ordinary linear regression (OLS) and solving the resultant models using exact diagonalization, we find a set of models which describe the energy functional on our $\mathcal{SS}$ accurately but whose eigenstates and spectra differ as a consequence of intruder states below an energy of 2eV. The sixteen different models in $\mathcal{MS}$ all regress with R$^2 >$  0.95. We ignore all models which have parameters which are zero within 95\% confidence intervals which we computed through a bootstrap estimate \textbf{ref}. Further, any models which have an average errorbar in eigenvalues $>$ 0.25 eV are ignored. This leaves us with just six models to further consider: the minimal model (Min), Min + $\bar{t}_\pi$, Min + $\bar{t}_{dz}$, Min + $\bar{t}_\pi, \bar{t}_{ds}$, Min + $\bar{t}_\pi, \bar{t}_{sz}$, Min + $\bar{t}_{ds}, \bar{t}_{dz}$. Within the latter five models there are intruder eigenstates, states which lie very far away from our sample set and suffer from large extrapolation errors. Intruder states were defined as any eigenstates with energy $< $ 2 eV with a Mahalanobis distance greater than the 90th percentile of all distances. The Mahalanobis distance is used in data science for outlier detection \textbf{ref} and tells us how far away an eigenstate is from $\mathcal{SS}$. Show in \textbf{Figure ?} are the identified intruder states among the six models above which in aggregate we will call the prior space $\mathcal{PS}$. These energies of these eigenstates are not constrained by $\mathcal{SS}$ because of our inability to target certain base states like $c^\dagger_{p_\pi} c_{d_\pi} |GS\rangle$ as discussed before. This excitation specifically is within our set of intruder states alongside others like $c^\dagger_{p_\pi} c_{d_{z^2}} |GS\rangle$.
Other methods like constrained DFT and CASSCF were used to converge these states but we were unsuccessful. 

In order to ensure that intruder states are not present within our final model, we enforce a prior that pushes intruder states above a chosen $E_{cut}$ = 2eV by altering the cost function for our regression. The choice of prior is built on both experimental knowledge and FN-DMC results. From experiment we know that any excitation which de-occupies a d-orbital should be $\sim$ 2 eV in energy. In fact all of our intruder states are excitations out of a d-orbital except for the higher energy intruder which is already nearly at 2eV and would not be affected too strongly by this choice of $E_{cut}$. While we cannot converge these missing base states using UKS, we can consider singles excitations on the UKS ground state which correspond to these excitations in order to get an upper bound for the FN-DMC energies of our missing base states. In FN-DMC the corresponding MO excited states are $> $4eV, and we believe even with orbital relaxation that the converged excited state should be $>$ 2eV in energy. The altered cost function takes the form:

\begin{equation}
\text{Cost} = \sum_{i \in \mathcal{SS}} (E_{eff,i} - E_{ab, i})^2 + \lambda \sum_{p \in \mathcal{PS}} \text{QHL}(2 - E_{eff,p})\ \text{QHL}(x) = \Theta(x)x^2
\end{equation}
where $E_{eff}$ is the energy of a state on our effective model, $\lambda>0$ is a parameter which can be varied, and QHL is a quadratic hinge loss with $\Theta$ the Heaviside step function. 
\section{Discussion}


\section{Conclusion}

\end{document}